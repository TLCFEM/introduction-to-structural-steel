\chapter{Preface}
\section{Standards}
This book refers to the following standards.
\begin{itemize}
\item \ASNZSACTION{}
\item \NZSSTEEL{}
\item \ASNZSSTEEL{}
\item \ASNZSPLATE{}
\item \ASNZSCOLD{}
\item \ASNZSWELD{}
\item \ASNZSFASTERNER{}
\item \ANSI{}
\end{itemize}
Due to various factors, this book is not aimed to cover everything in structural steel design. It is in general a good practice to have additional references at hand. Readers shall frequently refer to those standards for additional figures, tables, equations, etc.
\section{Additional Resources}
A number of distributors provide section properties. The most comprehensive ones are given here. Liberty Steel provides the manual for specifications of hot rolled structural steel sections manufactured in accordance with \ASNZSSTEEL{}.
\begin{itemize}
\item Liberty Steel Hot Rolled and Structural Steel Product\\This book is attached at the end of this document.
\end{itemize}
Fletcher Easysteel provides design charts of some section types.
\begin{itemize}
\item \href{run:./REF/STRUCTURAL.STEEL.PROPERTIES.DESIGN.CHARTS.BOOK.pdf}{Structural Steel Properties \& Design Charts Book}
\end{itemize}
AustubeMills provides design tables for hollow sections.
\begin{itemize}
\item \href{https://www.austubemills.com.au/en-au/resource-centre/resources/design-capacity-tables-for-structural-steel-hollow/}{Design Capacity Tables for Structural Steel Hollow Sections}\footnote{\url{https://www.austubemills.com.au/en-au/resource-centre/resources/design-capacity-tables-for-structural-steel-hollow/}}
\end{itemize}
The Australian Steel Institute provides design capacity tables for structural steel \citep{ASI2016}.
\begin{itemize}
\item \href{https://www.steel.org.au/resources/bookshop/products/design-capacity-tables-for-structural-steel,-vol/}{DESIGN CAPACITY TABLES FOR STRUCTURAL STEEL, VOL. 1: OPEN SECTIONS}
\end{itemize}

Readers shall frequently refer to those documents for properties that could be used in design.

The following YouTube channels provide engineering related contents of good quality.
\begin{itemize}
\item \href{https://www.youtube.com/channel/UCXAS_Ekkq0iFJ9dSUIkcAkw}{The Efficient Engineer}\footnote{\url{https://www.youtube.com/channel/UCXAS_Ekkq0iFJ9dSUIkcAkw}}
\end{itemize}
\section{To Use This Book}
If you are reading the digital copy of this book, you can check the additional links and worksheets provided.

Links coloured like this \href{https://www.canterbury.ac.nz/engineering/contact-us/people/gregory-macrae.html}{Gregory MacRae} are made available in the document. These are not necessary for the course, but they may be of interest to readers. Readers are welcome to report any invalid links in this book.

Worksheets prepared in \href{https://en.smath.com/view/SMathStudio/summary}{SMath Studio}\footnote{\url{https://en.smath.com/view/SMathStudio/summary}} for some examples are provided in a digital folder named as `WORKSHEET'. Readers may change parameters within the example if they wish. It must be noted that
\begin{itemize}
\item Assignments require handwritten submissions, but worksheets can be used to check the hand calculation.
\item The worksheets have been configured for the particular examples only. If readers wish to use the worksheets for actual design, they need to ensure that all relevant code clauses are satisfied.
\end{itemize}

Readers need to install it so that by clicking the `Worksheet' link, the corresponding worksheet can be automatically opened. Not all PDF readers support such a functionality. The feature--rich \href{https://www.tracker-software.com/product/pdf-xchange-viewer}{PDF-XChange Viewer}\footnote{\url{https://www.tracker-software.com/product/pdf-xchange-viewer}} is recommended. Please use the following one to check if the current one can work properly.
\begin{exmp}
\href{run:./WORKSHEET/CH00/EX0.TEST.sm}{Worksheet}
\textleftarrow~Click this link to test.
\end{exmp}