\chapter{Structural Steel Weld Specification and Implementation}
Fabricators generally only stock a few types of welding consumable so it is recommended that the following be used in practice.
\section*{A. Specification for Engineers for Construction Documents}
For welds to Grade 300 and 350 structural steel:
\begin{enumerate}
\item Weld metal shall be designated as $f_{uw}=\SI{490}{\mpa}$.
\item Welding consumables shall be matched with the steel types in accordance with AS/NZS 1554.1 (2014) Table 4.6.1(A).
\item All welding, including qualification of welding procedures, shall comply with AZ/NZS 1554.1 (2004; Am. 1, 2015; Am. 2, 2017).
\item Welds subject to earthquake loads or effects shall comply with AS/nZS 5131 (2016; Am., 2020) Section 7.5.17 and welding consumables shall have a Ship Classification Societies Grade 3 approval.
\item eld category specified for seismic structural applications, and for all construction to Construction Category (CC) 3, shall be SP to AS/NZS 1554.1 (2004; Am. 1, 2015; Am. 2, 2017).
\item Fabrication shall comply with the Construction Category (CC) in accordance with AS/NZS 5131 (2016; Am., 2020).
\item Weld inspection shall follow AZ/NZS 5131 (2016; Am., 2020) Appendix I.
\end{enumerate}
\section*{B. Fabricator Actions in Response to Above Specification}
\begin{enumerate}
\item Welding consumables shall be used within the welding parameter ranges specified by the manufacturer, and as per any relevant Ship Classification Societies approval.
\item Fabricator shall comply with the Construction Category (CC) specified for the job.
\end{enumerate}
\section*{C. Commentary}
\begin{enumerate}
\item The recommendations above were developed together with John Jones Steel and the HERA Welding Centre (June, 2021). They were developed to simplify weld specification.
\item Welding inspection can be performed as a combination of in-house welding inspection and by a third--party inspector if required. The standard is not specific as to who should perform the inspection. The type and quantity of inspection needs to be specified by the engineer. It can be performed by the fabricator (in-house), and/or by a third--party inspection company.
\item Fillet welds should be specified where possible because butt welds are generally more expensive.
\item Storage of welding consumables will comply with manufacturer's recommendations (to avoid cold cracking, porosity and other weld defects) as the fabricator meets the Construction Category (CC) requirements.
\end{enumerate}